\part{Introduction}


\chapter{Overview}

\noindent
This document is both a user's guide and a reference manual for
\Stan's probabilistic modeling language.  This introductory chapter
provides a high-level overview of \Stan. The remaining
parts of this document include a practically-oriented user's guide for
programming models and a detailed reference manual for \Stan's
modeling language and associated programs and data formats.

\section{Stan Home Page}

For links to up-to-date code, examples, manuals, bug reports,
feature requests, and everything else Stan related, see
the Stan home page:
%
\begin{quote}
\url{http://mc-stan.org/}
\end{quote}


\section{Stan Interfaces}

There are three interfaces for Stan that are supported as part of the
Stan project.  Models and their use are the same across the three
interfaces, and this manual is the modeling language manual for all
three interfaces.  All of the interfaces share initialization,
sampling and tuning controls, and roughly share posterior analysis
functionality.

The interfaces all provide getting-started guides, documentation, and
full source code.

\subsection{CmdStan}

CmdStan allows Stan to be run from the command line.  In some sense,
CmdStan is the reference implementation of Stan.  The CmdStan
documentation used to be part of this document, but is now its own
standalone document.  The CmdStan home page is
%
\begin{quote}
\url{http://mc-stan.org/cmdstan.html}
\end{quote}

\subsection{RStan}

RStan is the R interface to Stan.  RStan interfaces to Stan through
R's memory rather than just calling Stan from the outside, as in the
R2WinBUGS and R2jags interfaces on which it was modeled.  The RStan
home page is
%
\begin{quote}
\url{http://mc-stan.org/rstan.html}
\end{quote}

\subsection{PyStan}

PyStan is the Python interface to Stan.  Like RStan, it interfaces at
the Python memory level rather than calling Stan from the outside.
The PyStan home page is
%
\begin{quote}
\url{http://mc-stan.org/pystan.html}
\end{quote}

\subsubsection{MatlabStan}

MatlabStan is the MATLAB interface to Stan.  Unlike RStan and PyStan,
MatlabStan currently wraps aq CmdStan process.  The
MatlabStan home page is
%
\begin{quote}
\url{http://mc-stan.org/matlab-stan.html}
\end{quote}
%

\subsubsection{Stan.jl}

Stan.jl is the Julia interface to Stan.  Like MatlabStan, Stan.jl
wraps a CmdStan process.  The Stan.jl home page is
%
\begin{quote}
\url{http://mc-stan.org/julia-stan.html}
\end{quote}
%

\subsubsection{StataStan}

StataStan is the Stata interface to Stan.  Like MatlabStan, Stan.jl
wraps a CmdStan process.  The StataStan home page is
%
\begin{quote}
\url{http://mc-stan.org/stata-stan.html}
\end{quote}


\section{Stan Programs}

A \Stan program defines a statistical model through a conditional
probability function $p(\theta|y,x)$, where $\theta$ is a sequence of
modeled unknown values (e.g., model parameters, latent variables, missing
data, future predictions), $y$ is a sequence of modeled known
values, and $x$ is a sequence of unmodeled predictors and constants
(e.g., sizes, hyperparameters).

\Stan programs consist of variable type declarations and statements.
Variable types include constrained and unconstrained integer, scalar,
vector, and matrix types, as well as (multidimensional) arrays of
other types.  Variables are declared in blocks corresponding to the
variable's use: data, transformed data, parameter, transformed
parameter, or generated quantity.  Unconstrained local variables may
be declared within statement blocks.  

The transformed data, transformed parameter, and generated quantities
blocks contain statements defining the variables declared in their
blocks.  A special model block consists of statements defining the log
probability for the model.

Within the model block, \BUGS-style sampling notation may be used as
shorthand for incrementing an underlying log probability variable, the
value of which defines the log probability function.  The log
probability variable may also be accessed directly, allowing
user-defined probability functions and Jacobians of transforms.

\subsection{Variable Constraints}

Variable constraints are very important in Stan, particularly for
parameters.  For Stan to sample efficiently, any parameter values that
satisfy the constraints declared for the parameters must have support
in the model block (i.e., must have non-zero posterior density).

Constraints in the data and transformed data block are only used for
error checking data input and transforms.  Constraints in the
transformed parameters block must be satisfied the same way as
parameter constraints or sampling will devolve to a random walk or
fail.  Constraints in the generated quantities block must succeed or
sampling will be halted altogether because it is too late to reject
a draw at the point the generated quantities block is evaluated.

\subsection{Execution Order}

Statements in \Stan are interpreted imperatively, so their order
matters.  Atomic statements involve the assignment of a value to a
variable.  Sequences of statements (and optionally local variable
declarations) may be organized into a block.  \Stan also provides bounded
for-each loops of the sort used in \R and \BUGS.



\subsection{Probabilistic Programming Language}

Stan is an imperative probabilistic programming language.  It is an
instance of a domain-specific language, meaning that it was
developed for a specific domain, namely statistical inference.

Stan is a probabilistic programming language in the sense that a
random variable is a bona fide first-class object.  In Stan, variables
may be treated as random, and among the random variables, some are
observed and some are unknown and need to be estimated or used for
posterior predictive inference.  Observed random variables are
declared as data and unobserved random variables are declared as
parameters (including transformed parametrs, generated quantities, and
local variables depending on them).  For the unobserved random
variables, it is possible to sample them either marginally or jointly,
estimate their means and variance, or plug them in for downstream
posterior predictive inference.

Stan is an imperative language, like C or Fortran (and parts of \Cpp,
R, Python, and Java), in the sense that is based on assignment, loops,
conditionals, local variables, object-level function application, and
array-like data structures.  In contrast and/or complement, functional
languages typically allow higher-order functions and often allow
reflection of programming language features into the object language,
whereas pure functional languages remove assignment altogether.
Object-oriented languages introduce more general data types with
dynamic function dispatch.

Stan's language is Church-Turing complete
\cite{Church:1936,Turing:1936,HopcroftMotwani:2006}, in the same way
that C or R is.  That means any program that is computable on a Turing
machine (or in C) can be implemented in Stan (not necessarily easily,
of course).  All that is required for Turing completeness is loops,
conditionals, and arrays that can be dynamically (re)sized in a loop.

\section{Compiling and Running \Stan Programs}

A \Stan program is first translated to a \Cpp program by the \Stan
compiler \stanc, then the \Cpp program compiled to a self-contained
platform-specific executable.  \Stan can generate executables for
various flavors of Windows, Mac OS X, and Linux.%
%
\footnote{A \Stan program may also be compiled to a dynamically
  linkable object file for use in a higher-level scripting language
  such as \R or Python.}
%
Running the \Stan executable for a model first reads in and validates
the known values $y$ and $x$, then generates a sequence of
(non-independent) identically distributed samples $\theta^{(1)},
\theta^{(2)}, \ldots$, each of which has the marginal distribution
$p(\theta|y,x)$.


\section{Sampling}

For continuous parameters, \Stan uses Hamiltonian Monte Carlo (\HMC)
sampling \citep{Duane:1987, Neal:1994, Neal:2011}, a form of Markov
chain Monte Carlo (\MCMC) sampling \citep{Metropolis:1953}.  Stan does
not provide discrete sampling for parameters. Discrete observations
can be handled directly, but discrete parameters must be marginalized
out of the model.  \refchapter{mixture-modeling} and
\refchapter{latent-discrete} discuss how finite discrete parameters
can be summed out of models, leading to large efficiency gains versus
discrete parameter sampling.

\HMC accelerates both convergence to the stationary distribution and
subsequent parameter exploration by using the gradient of the log
probability function.  The unknown quantity vector $\theta$ is
interpreted as the position of a fictional particle.  Each iteration
generates a random momentum and simulates the path of the particle
with potential energy determined the (negative) log probability
function.  Hamilton's decomposition shows that the gradient of this
potential determines change in momentum and the momentum determines
the change in position.  These continuous changes over time are
approximated using the leapfrog algorithm, which breaks the time into
discrete steps which are easily simulated.  A Metropolis reject step
is then applied to correct for any simulation error and ensure
detailed balance of the resulting Markov chain transitions
\citep{Metropolis:1953, Hastings:1970}.

Basic Euclidean Hamiltonian Monte Carlo involves three ``tuning''
parameters to which its behavior is quite sensitive.  \Stan's samplers
allow these parameters to be set by hand or set automatically without
user intervention.

The first tuning parameter is the step size, measured in temporal
units (i.e., the discretization interval) of the Hamiltonian.  Stan
can be configured with a user-specified step size or it can estimate
an optimal step size during warmup using dual averaging
\citep{Nesterov:2009, Hoffman-Gelman:2011, Hoffman-Gelman:2014}.  In
either case, additional randomization may be applied to draw the step
size from an interval of possible step sizes \citep{Neal:2011}.

The second tuning parameter is the number of steps taken per
iteration, the product of which with the temporal step size determines
the total Hamiltonian simulation time.  \Stan can be set to use a
specified number of steps, or it can automatically adapt the number of
steps during sampling using the No-U-Turn (\NUTS) sampler
\citep{Hoffman-Gelman:2011, Hoffman-Gelman:2014}.

The third tuning parameter is a mass matrix for the fictional
particle.  \Stan can be configured to estimate a diagonal mass matrix
or a full mass matrix during warmup; Stan will support user-specified
mass matrices in the future.  Estimating a diagonal mass matrix
normalizes the scale of each element $\theta_k$ of the unknown
variable sequence $\theta$, whereas estimating a full mass matrix
accounts for both scaling and rotation,%
%
\footnote{These estimated mass matrices are global, meaning they are
  applied to every point in the parameter space being sampled.
  Riemann-manifold HMC generalizes this to allow the curvature implied
  by the mass matrix to vary by position.}
%
but is more memory and computation intensive per leapfrog step due to
the underlying matrix operations.

\subsection{Convergence Monitoring and Effective Sample Size}

Samples in a Markov chain are only drawn with the marginal
distribution $p(\theta|y,x)$ after the chain has converged to its
equilibrium distribution.  There are several methods to test whether
an \MCMC method has failed to converge; unfortunately, passing the
tests does not guarantee convergence.  The recommended method for
\Stan is to run multiple Markov chains, initialized randomly with a
diffuse set of initial parameter values, discard the warmup/adaptation
samples, then split the remainder of each chain in half and compute
the potential scale reduction statistic, $\hat{R}$
\citep{GelmanRubin:1992}.  If the result is not enough effective
samples, double the number of iterations and start again, including
rerunning warmup and everything.%
%
\footnote{Often a lack of effective samples is a result of not enough
  warmup iterations.  At most this rerunning strategy will consume
  about 50\% more cycles than guessing the correct number of
  iterations at the outset.}
  
When estimating a mean based on a sample of $M$ independent draws, the
estimation error is proportional to $1/\sqrt{M}$.  If the draws are
positively correlated, as they typically are when drawn using \MCMC
methods, the error is proportional to $1/\sqrt{\mbox{\sc n\_eff}}$,
where {\sc n\_eff} is the effective sample size.  Thus it is standard
practice to also monitor (an estimate of) the effective sample size
until it is large enough for the estimation or inference task at
hand.  

\subsection{Bayesian Inference and Monte Carlo Methods}

\Stan was developed to support full Bayesian inference.  Bayesian
inference is based in part on Bayes's rule,
\[
p(\theta|y,x) \propto p(y|\theta,x) \, p(\theta,x),
\]
which, in this unnormalized form, states that the posterior
probability $p(\theta|y,x)$ of parameters $\theta$ given data $y$ (and
constants $x$) is proportional (for fixed $y$ and $x$) to the
product of the likelihood function $p(y|\theta,x)$ and prior
$p(\theta,x)$.

For \Stan, Bayesian modeling involves coding the posterior probability
function up to a proportion, which Bayes's rule shows is equivalent to
modeling the product of the likelihood function and prior up to a
proportion.

Full Bayesian inference involves propagating the uncertainty in the
value of parameters $\theta$ modeled by the posterior $p(\theta|y,x)$.
This can be accomplished by basing inference on a sequence of samples
from the posterior using plug-in estimates for quantities of interest
such as posterior means, posterior intervals, predictions based on the
posterior such as event outcomes or the values of as yet unobserved
data.


\section{Variational Inference}

Stan implements a general, black-box form of variational inference,
which provide an approximate posterior distribution.  Variational
inference provides estimates of posterior means and uncertainty
througha parametric approximation of a posterior that is optimized for
its fit to the true posterior.  Variational inference can be much for
estimating posterior means and marginal variances than MCMC.  See
\refchapter{vi-algorithms} for more information.

\section{Optimization}

\Stan also supports optimization-based inference for models.  Given a
posterior $p(\theta|y)$, Stan can find the posterior mode $\theta^*$,
which is defined by
%
\[
\theta^{*} = \mbox{argmax}_{\theta} \ p(\theta|y).
\]
%
Here the notation $\mbox{argmax}_v \ f(v)$ is used to pick out the value
of $v$ at which $f(v)$ is maximized.

If the prior is uniform, the posterior mode corresponds to the maximum
likelihood estimate (MLE) of the parameters.  If the prior is not
uniform, the posterior mode is sometimes called the maximum a
posterior (MAP) estimate. 

For optimization, the Jacobian of any transforms induced by
constraints on variables are ignored.  It is more efficient in many
optimization problems to remove lower and upper bound constraints in
variable declarations and instead rely on rejection in the model
block to disallow out-of-support solutions.


\subsection{Inference with Point Estimates}

The estimate $\theta^{*}$ is a so-called ``point estimate,'' meaning
that it summarizes the posterior distribution by a single point,
rather than with a distribution.  Of course, a point estimate does
not, in and of itself, take into account estimation variance.
Posterior predictive inferences $p(\tilde{y} | y)$ can be made using
the posterior mode given data $y$ as $p(\tilde{y}|\theta^*)$, but they
are not Bayesian inferences, even if the model involves a prior,
because they do not take posterior uncertainty into account.  If the
posterior variance is low and the posterior mean is near the posterior
mode, inference with point estimates can be very similar to full
Bayesian inference.


\section{Variational Inference}

Stan also supports variational inference, an approximate Bayesian inference
technique \citep{Jordan:1999,Wainwright-Jordan:2008}. Variational inference has
had a tremendous impact on Bayesian computation, especially in the machine
learning community; it is typically faster than sampling techniques and can
scale to massive datasets \citep{Hoffman:2013}.

Variational inference approximates the posterior
$p(\theta \mid y)$ with a simple, parameterized distribution
$q(\theta \,;\, \phi)$. It matches the approximation to the
true posterior by minimizing the Kullback-Leibler divergence,
%
\[
  \phi^* = \argmin_\phi
  \KL{ q(\theta \,;\, \phi) }{ p(\theta \mid y) }.
\]
%
This converts Bayesian inference into an optimization problem with a
well-defined metric for convergence. Variational inference can provide orders of
magnitude faster convergence than sampling; the quality of the approximation
will vary from model to model. Note that variational inference is not a point
estimation technique; the result is a distribution that approximates the
posterior.

Stan implements Automatic Differentiation Variational Inference (ADVI), an
algorithm designed to leverage Stan's library of transformations and automatic
differentiation toolbox \citep{Kucukelbir:2015}. ADVI circumvents all of the
mathematics typically required to derive variational inference algorithms; it
works with any Stan model.



